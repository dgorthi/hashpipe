\documentclass[12pt]{article}
\usepackage[utf8]{inputenc}
\usepackage{listings}
\usepackage{xcolor}

\definecolor{codegreen}{rgb}{0,0.6,0}
\definecolor{codegray}{rgb}{0.5,0.5,0.5}
\definecolor{codepurple}{rgb}{0.58, 0, 0.82}
\definecolor{backcolor}{rgb}{0.95,0.95,0.92}

\lstloadlanguages{C}

\lstdefinestyle{Cstyle}{
    language=C,
    backgroundcolor=\color{white},
    commentstyle=\color{codegreen},
    keywordstyle=\color{blue},
    %numberstyle=\tiny\color{codegray},
    stringstyle=\color{codepurple},
    basicstyle=\ttfamily,
    breaklines=true,
    breakatwhitespace=false,
    breakautoindent=true,
    keepspaces=true,
    %numbers=left,
    %numbersep=5pt,
    showtabs=false,
    tabsize=2
}

\lstset{style=Cstyle}
\def\clst{\lstinline[basicstyle=\ttfamily,breaklines=true,language=C]}

\begin{document}

\title{List of Hashpipe Structures} \author{Deepthi Gorthi}
\maketitle

\section{Thread Structures and Functions}

\subsection{\clst{hashpipe_thread_desc_t}}

\begin{lstlisting}
        struct hashpipe_thread_desc {
          const char * name;
          const char * skey;
          initfunc_t init;
          runfunc_t run;
          databuf_desc_t ibuf_desc;
          databuf_desc_t obuf_desc;
        };
\end{lstlisting}

The \clst{hashpipe_thread_desc} structure is used to store metadata 
describing a hashpipe thread.  Typically a hashpipe plugin will 
define one of these hashpipe thread descriptors per hashpipe thread.  

\begin{enumerate}
\item \clst{const char *name}: String containing the thread name. Hashpipe threads are 
identified by their names which need to be registered with 
\clst{register_hashpipe_thread()}. This is used to match command 
line thread specifiers to thread metadata so that the pipeline can be 
constructed as specified on the command line.

\item \clst{const char *skey}: String containing the thread's status 
buffer ``status" key. It is typically 8 characters or less, uppercase and 
ends with ``STAT". If it is \clst{non-NULL} and non-empty, \clst{HASHPIPE} 
will automatically store/update this key in the status buffer with the 
thread's status at initialization (``init") and exit (``exit").

\item \clst{initfunc_t init}: Pointer to thread's initialization function. The thread 
initialization function can be null if no special initialization is needed.  
If provided, it must point to a function with the following signature:

\clst{int my_thread_init_funtion(hashpipe_thread_args_t *args) }

\item \clst{runfunc_t run}: Pointer to thread's run function. The thread run 
function must have the following signature:

\clst{void my_thread_run_funtion(hashpipe_thread_args_t *args) }

\item \clst{ibuf}: Structure describing thread's input data buffer (if any)

\item \clst{obuf}: Structure describing thread's output data buffer (if any). The 
data buffer description structure used for ibuf and obuf currently contains 
one function pointer:

create - A pointer to a function that creates the data buffer.

Future \clst{HASHPIPE} versions may introduce additional data buffer fields. 
\clst{ibuf.create} should be \clst{ NULL} for input-only threads and 
\clst{obuf.create} should be \clst{ NULL} for output-only threads.  
Having both \clst{ibuf.create} and \clst{ obuf.create} set to \clst{ NULL} is 
invalid and the thread will not be used. The create function must have the 
following signature:

\begin{lstlisting}
hashpipe_databuf_t *my_create_function(           int instance_id, int databuf_id)
\end{lstlisting}
\end{enumerate}

\subsection{\clst{hashpipe_thread_args}}

This structure passed (via a pointer) to the application's thread
initialization and run functions.  The `user\_data' field can be used to pass
info from the init function to the run function.

\begin{enumerate}

\item \clst{hashpipe_thread_desc_t *thread_desc}
\item \clst{int instance_id}
\item \clst{int input_buffer}
\item \clst{int output_buffer}
\item \clst{unsigned int cpu_mask}: 0 means use inherited
\item \clst{int finished}
\item \clst{pthread_cond_t finished_c}
\item \clst{pthread_mutex_t finished_m}
\item \clst{hashpipe_status_t st}
\item \clst{hashpipe_databuf_t *ibuf}
\item \clst{hashpipe_databuf_t *obuf}
\item \clst{void *user_data}
\end{enumerate}

\subsection{Useful Functions}

\begin{enumerate}
\item\clst{int run_threads()}: Function threads used to determine 
whether to keep running.

\item\clst{register_hashpipe_thread (hashpipe_thread_desc_t *ptm)}: 
Function should be used by pipeline plugins to register threads with 
the pipeline executable.

\item\clst{hashpipe_thread_desc_t *find_hashpipe_thread(char *name)}:
This function can be used to find hashpipe threads by name.  
It is generally used only by the hashpipe executable.  Returns a 
pointer to its \clst{hashpipe_thread_desc_t} structure or 
\clst{NULL} if a test with the given name is not found. Names are 
case sensitive.

\item\clst{void list_hashpipe_threads(FILE *f)}: List all known hashpipe threads 
to file pointed to by the file pointer.

\item\clst{unsigned int get_cpu_affinity()}: Get the CPU affinity of calling 
thread.

\end{enumerate}

\section{Data Buffer Structures and Functions}

\subsection{\clst{hashpipe_databuf_t}}

\begin{enumerate}
\item \clst{char data_type[64]}: Type of data in the buffer
\item \clst{size_t header_size}: Size of each block header in bytes 
\item \clst{size_t block_size}: Size of each data block in bytes.
\item \clst{int n_block}: Number of data blocks in buffer
\item \clst{int shmid}: ID of this shared memory segment
\item \clst{int semid}: ID of locking semaphore set
\end{enumerate}

\subsection{Associated functions}

\begin{enumerate}
\item {\bf Key}:

\clst{key_t hashpipe_databuf_key()}

Get the base key to use for {\bf all} hashpipe databufs.  The base key is
obtained by calling the ftok function, using the value of 
{\tt \$HASHPIPE\_KEYFILE}, if defined, or {\tt \$HOME} from the environment or, 
if {\tt \$HOME} is not defined, by using {\tt \/tmp}. By default 
(i.e. no {\tt HASHPIPE\_KEYFILE} in the environment), this will create and 
connect to a user specific set of shared memory buffers (provided {\tt \$HOME} 
exists in the environment), but if desired users can connect to any other set 
of memory buffers by setting {\tt HASHPIPE\_KEYFILE} appropriately.

\item {\bf Create Databuf}:

\begin{lstlisting}
hashpipe_databuf_t *hashpipe_databuf_create(int instance_id,int databuf_id, size_t header_size, size_t block_size, int n_block)
\end{lstlisting}

Create a new shared mem area with given params.  Returns pointer to the new
area on success, or \clst{NULL} on error.  Returns error if an existing shmem area
exists with the given shmid and different sizing parameters. 

\item {\bf Get Databuf}:

\clst{hashpipe_databuf_t *hashpipe_databuf_attach(int   instance_id, int databuf_id)}

Return a pointer to a existing shmem segment with given id. Returns error 
if segment does not exist 

\item {\bf Detach Databuf}:

\clst{int hashpipe_databuf_detach(hashpipe_databuf_t *d)}

Detach from shared mem segment

\item {\bf Clear Databuf}:

\clst{void hashpipe_databuf_clear(hashpipe_databuf_t *d)}

\item {\bf Reset Pointer location}:
\begin{lstlisting}
char *hashpipe_databuf_data(   hashpipe_databuf_t *d, int block_id)
\end{lstlisting}

Returns pointer to the beginning of the given data block.

\item {\bf Get lock status}:

\begin{lstlisting}
int hashpipe_databuf_block_status(hashpipe_databuf_t *d, int block_id)

int hashpipe_databuf_total_status(hashpipe_databuf_t *d)

uint64_t hashpipe_databuf_total_mask(hashpipe_databuf_t *d)
\end{lstlisting}

Returns lock status for given block\_id, or total for whole array.

\item {\bf Locking functions}:

\begin{lstlisting}
int hashpipe_databuf_wait_filled(hashpipe_databuf_t *d, int block_id)

int hashpipe_databuf_busywait_filled(hashpipe_databuf_t *d, int block_id)

int hashpipe_databuf_set_filled(hashpipe_databuf_t *d, int block_id)

int hashpipe_databuf_wait_free(hashpipe_databuf_t *d, int block_id)

int hashpipe_databuf_busywait_free(hashpipe_databuf_t *d, int block_id)

int hashpipe_databuf_set_free( hashpipe_databuf_t *d, int block_id)
\end{lstlisting}

Databuf locking functions.  Each block in the buffer
can be marked as free or filled.  The "wait" functions
block (i.e. sleep) until the specified state happens.
The "busywait" functions busy-wait (i.e. do NOT sleep)
until the specified state happens.  The "set" functions
put the buffer in the specified state, returning error if
it is already in that state.

\end{enumerate}

\section{Status Handling}
Routines dealing with the hashpipe status shared memory
segment.  Info is passed through this segment using
a FITS-like keyword=value syntax.

\begin{centering}
\begin{lstlisting}
/* Structure describes status memory area */
typedef struct {
  /* Instance ID of this status buffer (DO NOT SET/CHANGE!) */
    int instance_id;

  /* Shared memory segment id */
    int shmid;

  /* POSIX semaphore descriptor for locking */
    sem_t *lock;

  /* Pointer to data area */
    char *buf; 
} hashpipe_status_t;
\end{lstlisting}
\end{centering}

\subsection*{Status handling functions}
\begin{enumerate}

\item Semaphore name:
Stores the hashpipe status (POSIX) semaphore name in semid buffer of length
size.  Returns 0 (no error) if semaphore name fit in given size, returns 1
if semaphore name is truncated.

The hashpipe status semaphore name is {\tt \$HASHPIPE\_STATUS\_SEMNAME} (if defined
in the environment) or {\tt \${HASHPIPE\_KEYFILE}\_hashpipe\_status} (if defined in
the environment) or {\tt \${HOME}\_hashpipe\_status} (if defined in the environment)
or {\tt "/tmp\_hashpipe\_status"} (global fallback).  Any slashes after the leading
slash are converted to underscores.

\clst{int hashpipe_status_semname(int instance_id, char * semid, size_t size); }

\item Instance ID number:

Returns non-zero if the status buffer for instance\_id already exists.

\clst{int hashpipe_status_exists(int instance_id);}

\item Attach buffer:

Return a pointer to the status shared mem area,
creating it if it doesn't exist.  Attaches/creates
lock semaphore as well.  Returns nonzero on error.

\clst{int hashpipe_status_attach(int instance_id, hashpipe_status_t *s); }


\item Detach buffer: Detach from shared mem segment

\clst{int hashpipe_status_detach(hashpipe_status_t *s);}

\item Locking/Unlocking status buffer:

Lock/unlock the status buffer.  hashpipe\_status\_lock() will sleep while
waiting for the buffer to become unlocked.  hashpipe\_status\_lock\_busywait
will busy-wait while waiting for the buffer to become unlocked.  Return
non-zero on errors.

\begin{centering}
\begin{lstlisting}

int hashpipe_status_lock(hashpipe_status_t *s);

int hashpipe_status_lock_busywait(hashpipe_status_t *s);

int hashpipe_status_unlock(hashpipe_status_t *s);

\end{lstlisting}
\end{centering}

\item Check formatting:

Check the buffer for appropriate formatting (existence of {\tt "END"}).
If not found, zero it out and add {\tt END}.

\clst{void hashpipe_status_chkinit(hashpipe_status_t *s);}

\item Clear: Clear out whole buffer

\clst{void hashpipe_status_clear(hashpipe_status_t *s);}

\item Thread safe lock/unlock:

Thread-safe lock/unlock macros for status buffer used to ensure that the
status buffer is not left in a locked state.  Each hashpipe\_status\_lock\_safe
or hashpipe\_status\_lock\_busywait\_safe must be paired with a
hashpipe\_status\_unlock\_safe in the same function and at the same lexical
nesting level.  See "man pthread\_cleanup\_push" for more details.

\paragraph{Note} This header file does not include pthread.h where pthread\_cleanup\_push
and pthread\_cleanup\_pop are defined.  Users of the macros defined here
must explicitly include pthread.h themselves.

\begin{centering}
\begin{lstlisting}

#define hashpipe_status_lock_safe(s) \
    pthread_cleanup_push((void (*)(void *))hashpipe_status_unlock, s); \
    hashpipe_status_lock(s);

#define hashpipe_status_lock_busywait_safe(s) \
    pthread_cleanup_push((void (*)(void *))hashpipe_status_unlock, s); \
    hashpipe_status_lock_busywait(s);

#define hashpipe_status_unlock_safe(s) \
    hashpipe_status_unlock(s); \
    pthread_cleanup_pop(0);
\end{lstlisting}
\end{centering}

\end{enumerate}

\section{Error Handling}

\subsection{Exit codes}
\vspace{1cm}
\begin{center}
\begin{tabular}{|c|c|c|}
\hline
{\tt HASHPIPE\_OK}          &  0 & Everything is great\\[0.25cm] 
\hline
{\tt HASHPIPE\_TIMEOUT}     &  1 & Call timed out \\[0.25cm]
\hline
{\tt HASHPIPE\_ERR\_GEN}    & -1 & Super non-informative \\[0.25cm]
\hline
{\tt HASHPIPE\_ERR\_SYS}    & -2 & Failed system call \\[0.25cm]
\hline
{\tt HASHPIPE\_ERR\_PARAM}  & -3 & Parameter out of range \\[0.25cm]
\hline
{\tt HASHPIPE\_ERR\_KEY}    & -4 & Requested key doesn't exist \\[0.25cm]
\hline
{\tt HASHPIPE\_ERR\_PACKET} & -5 & Unexpected packet size \\[0.25cm]
\hline
\end{tabular}
\end{center}

\subsection{Error logging}
\begin{enumerate}
\item log error: Call this to log an error message.

\clst{void hashpipe_error(const char *name, const char *msg, ..);}

\item log warning: Call this to log an warning message.

\clst{void hashpipe_warn(const char *name, const char *msg, ...);}

\item log info: Call this to log an informational message.

\clst{void hashpipe_info(const char *name, const char *msg, ...);}


\end{document}
