\documentclass[12pt]{article}
\usepackage[utf8]{inputenc}

\begin{document}

\title{Tutorial ReadMe} \author{Deepthi Gorthi}
\maketitle

Hashpipe is a collection of useful structures and functions that will help 
you create a pipeline for high speed data collection from an ethernet 
port to say, a file system. This module is by no means a substitute to lack of 
knowledge about what goes into creating a high speed data pipeline or 
C programming. It is just a tool that will immensely aid your attempt if you 
already know what needs to be done.

Don't be discouraged though! It is easy to learn if your data rate is not 
very high (~1Gbps at the time of writing).

\paragraph{Barebones Hashpipe} Let's start on the axiom that you need a 
multi-threaded process. At the very least your pipeline should have a thread that 
takes data from a socket and feeds into a shared memory, another thread that 
reads this data, processes it and feeds into another shared memory and a final 
thread that writes it to disk. Depending on the datarate you are catering to, 
you can omit one or more of these steps, the limit being using a serial code 
that is not multi-threaded.

This tutorial contains just those three threads and a few additional ones to 
link up all of them; to perform the simple task of adding two numbers. 
Here is map:

\begin{enumerate}
\item {\tt recv\_data\_thread.c}: In an actual application this thread takes in 
high speed data from a socket. For the purposes of this tutorial, it asks the 
user to input two numbers and stores it in a shared memory.

\item {\tt process\_data\_thread.c}: This thread adds the two numbers and puts
the sum in a second shared memory.

\item {\tt output\_data\_thread.c}: This reads the second shared memory and print 
the sum to the terminal screen.

\item {\tt databuf.c}: This program defines the size of the ring buffer you need 
for this process. The ring buffer ({\tt hashpipe\_databuf\_t} structure) needs a 
header size ($>96$~bytes), block size and number of blocks in the ring buffer.

In principle, this file can be much more complicated with your custom structures 
defining the ordering of your data.

\item {\tt databuf.h}: Header file containing the structures you want to define. 
In our case this just contains the function that creates our shared buffers.

\item {\tt Makefile}: To link all the above threads, header files, hashpipe etc. 
and compile them. {\bf NB}: Make sure you {\tt make all} and {\tt make install} 
for the plugin to be callable. 

\item {\tt init.sh}: Calls the plugin.

\end{enumerate}
\end{document}
